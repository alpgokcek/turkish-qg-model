\documentclass[10pt]{article}

\usepackage[table,xcdraw]{xcolor}
\usepackage{graphicx}
\usepackage[left=2.54cm, right=2.54cm, top=3.18cm, bottom=2.54cm]{geometry}
\usepackage{fontspec}
\usepackage{titlesec}
\usepackage{setspace}
\usepackage{enumitem}
\usepackage{colortbl}
\usepackage{tabularx}
\usepackage{geometry}
\usepackage{float}
\usepackage{fancyhdr}
\pagestyle{fancy}
\defaultfontfeatures{Mapping=tex-text,Scale=MatchLowercase}
\setmainfont{Arial}

\definecolor{azure}{RGB}{25,60,102}
\definecolor{darkazure}{RGB}{37,95,166}
\definecolor{danube}{RGB}{106,157,212}


\titleformat{\section}
{\large\bfseries\color{darkazure}}
{}{0.0001em}{}
\renewcommand{\headrulewidth}{0pt}

\rhead{
	\vspace*{-3em}
	\includegraphics[height=1.45cm]{../assets/mef_logo.png}
}


\begin{document}
	\arrayrulecolor{danube}
	

	
	\begin{spacing}{1.1}
	{\color{azure}
		\bfseries
		\huge{
			\noindent
			COMP492 SENIOR DESIGN PROJECT II \hspace{0pt} \\
			PROPOSAL FORM \par
		}
		\large{
			\noindent
			DEPARTMENT OF COMPUTER ENGINEERING \newline \par
		}
	}
	\end{spacing}
	\vspace{-2em}
	\section{Project Name}
	
	Turkish Question Generation Model
	

	\section{Project Summary (Abstract)}
	
	Our goal in this project is to develop a Turkish Question Generation System using available online sources such as Wikipedia. Question Generation Systems are capable of generate various logical questions from the given text input. QG Systems are prevalent in several computer applications such as chatbots, automated grading systems etc. Development of such a system is particularly important since there are not any examples of online tools and available datasets in Turkish. However, QG research has already reached a notable level in other languages such as English and there are available datasets for this task. For instance, SQuAD by Stanford University is one of the well- known datasets used for this task in English. \newline \par
	
	In this project, we are aiming to create the first Turkish Question Generation Dataset of moderate size. This step is required for developing learning-based solutions that we’re going to apply in this project. We plan to apply different machine learning based techniques for questions of generating varying forms (Who-What-When etc.) and assess their performances on the dataset we’ll gather. \newline \par
	
	Furthermore, in the next semester, we are going to explore the applicability of deep learning models on this topic. Outcome of this research will be integrated into a larger project which focuses on Automated Question Answering system for the online lectures in the Turkish Language. (This is a joint project with EE Dept. under the supervision of Ebru Arisoy Saraclar)
	
	\section{Keywords}
	
	Automatic speech recognition, Natural Language Processing, Deep Learning, Sentiment Analysis, Education Technology, MOOC (Massive Open Online Courses). \newline \par
	
	\section{Hardware and Software Requirements}
	
	\begin{itemize}[label=\textcolor{darkazure}{\Large\textbullet}]
		\item GPU Optimized Server. Will be used for developing deep learning models, storing large amounts of data and as a collaborative environment.
		
		\item UNIX Based Environment(s). Given the design and integration of de-facto programming languages, frameworks, tools with Unix based operating systems, along with the flexibility and low-level tools it has to offer, we concluded that UNIX based operating systems will be most suitable for the development process.
	\end{itemize}
	
	
	\section{Project Tasks, Time Plan and Deliverables}
	\begin{center}
		\begin{table}[H]
			\begin{tabularx}{\textwidth}{|>{\raggedright\arraybackslash}m{28mm}|m{20mm}|m{28mm}|m{28mm}|m{39mm}|}					
				\rowcolor[RGB]{215,229,244}
				\hline
				\multicolumn{1}{|>{\centering\arraybackslash}m{28mm}|}{\textbf{Task}} 
				& \multicolumn{1}{>{\centering\arraybackslash}m{20mm}|}{\textbf{Start \& Due \newline Dates}} 
				& \multicolumn{1}{>{\centering\arraybackslash}m{28mm}|}{\textbf{Deliverable}} 
				& \multicolumn{1}{>{\centering\arraybackslash}m{28mm}|}{\textbf{Evaluation Criteria}} 
				& \multicolumn{1}{>{\centering\arraybackslash}m{39mm}|}{\textbf{Objective}}\\
				Project Proposal & 09/10/2020\newline 23/10/2020 & Proposal with clear goals. & Readable, clear & Expressing our intent for the research project and our tasks for this semester. \\ \hline  
				Development\newline Environment\newline Preparation & 19/10/2020\newline10/11/2020 & A working environment & Installment of necessary tools, programming languages etc. & Embracing the concepts of the operating system, programming language, frameworks and best practices to be used during the development stages \\ \hline  
				Literature Review and data collection & 19/10/2020\newline30/11/2020 & A collection of related papers and a training dataset for QG. & Moderate size dataset and collection of recent publications & Learning the foundations of the learning-based methods, developing insight about the concepts. Learning state-of-the art in this research area. Collecting the first Turkish QG dataset. \\ \hline  
				Question Generation System Prototype & 01/11/2020\newline01/01/2021 & Prototype System & A working prototype with limited capabilities & Generating questions of different forms from a given text. \\ \hline  
				Final Report & 01/01/2021\newline08/01/2021 & Final Report & Complete, readable, formatted & Covering the work done in detail. \\
				\hline  
			\end{tabularx}
		\end{table}
	\end{center}
	
	
	\section{Project Team and Authority Information}
	\begin{table}[H]
		\begin{tabularx}{\textwidth}{|>{\columncolor[RGB]{215,229,244}\hsize=0.33\hsize}X|
				>{\hsize=.67\hsize}X|} 	
			
			\hline
			\textbf{Proposal Date} & 08/03/2020 \\ \hline
			\textbf{Academic Term of Project Delivery} & 2020-2021, Spring semester \\ \hline
			\textbf{Project Team Members} & Alp Gokcek, \#041701014, Computer Engineering (Major) \newline Erdal Sidal Dogan, \#041701076, Computer Engineering (Major)  \\ \hline
			\textbf{Advisor(s)} & Seniz Demir \\ \hline
		\end{tabularx}
	\end{table}
\end{document}